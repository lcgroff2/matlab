\documentclass[12pt]{article}
\usepackage{graphicx,cite}
\usepackage{times}
\usepackage[letterpaper]{geometry}
\usepackage{amsmath}
%\usepackage{doublespace}
\geometry{text={6.25in,8.5in}, top=1in,
left=1in}

\newcommand{\rang}      {\AA^{-1}}      %reciprocal angstroms


\def\bra#1{\langle#1\vert}              % \bra{stuff} gives <stuff|
\def\ket#1{\vert#1\/\rangle}            % \ket{stuff} gives |stuff>
\def\vev#1{\langle{#1}\rangle}          % \vev{stuff} gives <stuff>
\def\Ham{{\mathcal H}}
\def\D{{\partial}}
\def\DD#1#2{\frac{\partial#1}{\partial#2}}
\def\EE#1{\times 10^{#1}}


\title{Modeling Exciton Diffusion in Nanoparticles}
\author{J. McNeill}
\date{March 7, 2012}

\begin{document}
\maketitle


\noindent {\Large \bf Overview} \vskip 10 pt

\noindent First, let's get some terms out of the way.  By {\em
  exciton}, here we are referring to a localized molecular excitation
on a relatively short segment of conjugated polymer.  This type of
molecular exciton is very different from typical (Wannier) excitons in
inorganic semiconductors (which are considered electron-hole pairs and
are much larger, are much more weakly bound, and are typically only
stable at very low temperatures, 4 K or less).  Heeger (Nobel
prize-winner) has theorized that conjugated polymers are very much
like inorganic semiconductors, and that excitons in CP's are also
weakly-bound electron-hole pairs. However, there is considerable
evidence that this view is wrong, and he is in the minority on this
subject.  The type of exciton we are referring to for conjugated
polymers is the small or {\em molecular exciton} (sometimes called a
Frenkel exciton, but I prefer not to, see below), which consists of a
molecule in the $\pi$* excited state and the polarization of the
surrounding polymer molecules.  Strong coupling between the transition
dipoles of nearby chromophores can give rise to collective excitations
called {\em Frenkel excitons}, which can in principle be many times
larger than a single chromophore.  Frenkel excitons often exhibit an
energy shift or energy splitting, referred to as the Davydov
splitting.  However, for the sake of simplicity, here we will assume
that the excitons are small--much smaller than the nanoparticle, and
we will assume that Frenkel exciton behavior (such as collective
excitation, Davydov splitting and coherent, ultrafast energy transfer)
is either unimportant or is partially taken into account in the
exciton diffusion theory, {\it i.e.}, ultrafast/coherent transport
results in a somewhat larger exciton diffusion length.

Another quick note about excitons.  Wikipedia, and many other places,
define an exciton as a bound state of an electron and hole which are
attracted to each other by a Coulomb force.  While this is correct for
Wannier excitons, it is {\bf very wrong for molecular/Frenkel
  excitons}, since the picture of a bound electron-hole pair typically
doesn't make sense for molecules.  For example, if we plug in a
typical molecular exciton binding energy of 0.5 eV and a typical
dielectric constant of 4, this gives an orbit of less than 1 \AA, much
smaller than the molecule. This is physically absurd, thus the
electron-hole pair picture doesn't make sense for molecular/Frenkel
excitons.  Additional information about Frenkel excitons and the
Frenkel-exciton Hamiltonian, which is based on coupling of transition
dipoles, is given in a paper by Kasha and El-Bayoumi, ``The Exciton
Model in Molecular Spectroscopy.''

By {\em chromophore}, I am referring to a small segment of the polymer
chain, typically 3-8 polymer repeat units in length.  The exciton can
``hop'' more or less randomly from chromophore to chromophore
repeatedly during the lifetime of the exciton.  Because of the
similarity of this behavior to the phenomenon of diffusion, this
process is called {\em exciton diffusion}, and it occurs by either
overlap of molecular orbital wavefunctions (often referred to as
Dexter transfer), by F\" orster transfer, or by coherent ultrafast
energy transfer.  All are known to occur, likely simultaneously, and
it can be difficult to experimentally determine which process or
processes are dominant. The question of which mechanism dominates
depends on a variety of system physical parameters such as chain
conformation, disorder, interchromophore distances, transition dipole
moments, temperature, and the electronic states (molecular orbitals)
involved.  We might want to design some experiments (low T
experiments, maybe some time-resolved polarization anisotropy) to
investigate these issues, including such basic issues as effective
chromophore size and the length scales and time scales of individual
hopping events, but for now, in this document, we will not worry
further about the detailed mechanism for exciton hopping/diffusion.

In the standard picture of exciton diffusion theory, at any given
moment in time, an exciton can either hop to another site (diffuse),
decay radiatively (emit a photon), or decay non-radiatively (there are
many possibly non-radiative pathways, including internal conversion to
vibrations, electron transfer to make a charge-separated state, or
intersystem crossing to make a triplet state).  In a disordered
material there are additional complications, because there will be a
range of chromophore energies, and exciton hopping will tend to
proceed from higher energy chromophores (ones with a larger
$\pi$-$\pi$* gap) to lower energy chromophores, eventually becoming
trapped at low-energy chromophores.  This tends to result in the
relatively large gap between excitation and emission (as compared to
dyes) typically seen for conjugated polymers, particularly for the
films and nanoparticles, and other phenonomena, such as a progressive
red-shifting of the emission on the few picosecond timescale (seen in
fluorescence upconversion experiments) and a related slowing-down of
the rate of the energy diffusion on the picosecond timescale, as the
typical exciton energy decreases and the fraction of energetically
accessible chromophores decreases--as an exciton gets redder, there
are fewer and chromophores that are still redder and can thus accept
the exciton. We will also ignore these phenomena in our model, for
now, but it is worthwhile to note them.

If we add an energy acceptor (dye/quencher), then occasionally an
exciton will decay by transferring its energy to the acceptor, likely
via a F\" orster transfer (FRET) mechanism.  Since the rate of FRET
depends on the inverse sixth power of the distance, and the dyes and
excitons are distributed more or less randomly, this gives rise to a
large range of energy transfer rates.  This process has been examined
in some detail for a number of years, dating back to early work in the
60's and 70's by Powell (look in the EndNote database) on doped
anthracene crystals, and more recently by theorists such as Nakanishi
and experimentalists such as Andy Monkman.  Many details haven't been
figured out, and it's still not clear the relative roles of exciton
(chromophore) size, Dexter vs. Forster transfer, collective
excitations (coherent energy transfer, Frenkel exciton picture),
energetic disorder, etc.

The conjugated polymer nanoparticles give us another window into these
energy transfer phenomena.  By controlling the particle size, we can
control the length scale over which energy transfer occurs.  We also
have some control of polymer phase (e.g., the beta phase in PFO).
Also, it is much easier to collect accurate and reproducible
fluorescence quantum yield results and time-resolved fluorescence
(TCSPC) on the nanoparticle dispersions than it is to collect data on
thin films.  Most prior work in this area is on thin films.  Thin
films tend to be difficult to reproduce, and there are interface
effects and waveguiding effects that are hard to quantify, control and
model.

We have already seen some evidence that energy transfer efficiency
depends on particle size.  This is exciting, and provides the
motivation to study these effects further.  We plan on conducting a
number of TCSPC studies to look at energy transfer for a range of
particle sizes and doping concentrations.  By careful analysis of the
TCSPC traces, we hope to find direct evidence of ``distributed''
energy transfer rates (a range of rates due to the random distribution
of dyes and range of Forster distances).  This by itself should be
publishable in a P-Chem journal.  In addition, we could follow this up
with time-resolved polarization anisotropy (straightforward, just add
polarizers to the TCSPC setup), single molecule measurements, and low
temperature measurements.  We could even later add femtosecond
time-resolved fluorescence upconversion measurements (if you're
interested, I could point you to some papers on how this is done).  All
of these results should give us an unprecedented level of detail in
determining the fundamental physics of exciton interactions and energy
transfer in CPs and other molecular semiconductors.  This is a very
important practical and fundamental question, very relevant to
optimizing our nanoparticles for PDT and other applications, and even
for other applications such as polymer displays and solar cells.

We are way ahead of the competition in that we already have flexible,
sophisticated, easy-to-use code for an exciton diffusion and energy
transfer model that works for nanoparticles (trivial to extend to bulk
films--just use larger nanoparticles), and we have the nanoparticle
system and fluorescence quantum yield and lifetime data.  This should
lead to a series of publications in physical chemistry journals and
possibly high profile publications as we progress.

\vskip 24 pt

\noindent {\Large \bf 3D Random Walk Model: Monte Carlo
  Approach} \vskip 10pt

\noindent We are modeling the combined processes of exciton diffusion,
fluorescence, non-radiative relaxation, and energy transfer using a 3D
random walk simulation employing a Monte Carlo approach--simulating
random exciton trajectories using a random number generator.
Previously, in Changfeng's dye-doped nanoparticle paper and papers by
Jiangbo, we treated the motion of the excitons as a random walk on a
3D cubic lattice.  There were some shortcomings with that code when it
came to simulating our recent (2012) experiments, so I started writing
a different version, and in the process found a bug in the old code
that is difficult to fix (and introduces some error, but not enough to
invalidate our previous results). Therefore we are switching to the
new code, based on a slightly different picture of diffusion and
Brownian motion.  I have verified that this code is more accurate than
the previous code, and it has been extensively tested.  Here is the
approach taken for the new code.  According to Brownian motion theory,
if we have a particle with a diffusion coefficient $D$, initially at
position $x_0$ at time $t_0$, the particle will, at a later time $t$,
be at a random position given by a Gaussian probability distribution,
%
$$
p(x) = \frac{1}{\sigma \sqrt{2\pi}} e^{-(x-x_0)^2/2 \sigma^2},
$$
%
where the standard deviation $\sigma$ is given by
%
$$
\sigma = \sqrt{2 D (t-t_0)}.
$$
%
It can be shown that trajectories that the above expressions reproduce
the correct 1D Brownian motion behavior for the mean square displacement
$\vev{x^2(t-t^\prime)}$,
%
$$
\vev{x^2(t-t^\prime)} = 2 D t
$$
%
Thus, in our Monte Carlo approach, if we have a time step $\Delta t$,
we can propagate the particle (exciton) at each time step by
generating a Gaussian-distributed random numbers with a standard deviation of
$\sigma$.  In {\sc matlab} this is achieved by the {\tt randn}
function--you are encouraged to familiarize yourself with the
function.  For example, plotting {\tt plot(3*randn([1 100]))} will
plot gaussian-distributed random numbers with a standard deviation of
3.  We also need to add the possibilities of decay (fluorescence) and
energy transfer to our random walk, and we need to calculate the
diffusion constant based on the lifetime and diffusion length.

In the exciton diffusion experiments, it is difficult to determine the
diffusion constant directly, so usually what is determined and
reported in the literature is an exciton {\em diffusion length},
$L_D$, typically in the range of about 4 nm to 15 nm for conjugated
polymers.  Exciton diffusion lengths as high as hundreds of nm are
reported for highly crystalline anthracene at low temperature.  Be
careful when looking in the literature--there are lots of improbably
large exciton diffusion lengths reported, likely due to systematic
error, sloppy experiments, overinterpreted experiments, etc.  For
example, Gregg et al. concluded that thin films of a perylene
derivative (PPEI) yielded large (2.5 $\mu$m!) exciton diffusion
lengths, on the basis of experiments in which a film was excited from
one side and the emission of an acceptor layer on the other side of
the film was measured (B. Gregg 1997 JPC-B).  However, disorder
(pinholes, etc) in the PPEI film, intermixing of the layers, and
perhaps some waveguiding were likely responsible for the observed
acceptor emission--other experiments with uniformly dye-doped films
and using NSOM were unable to reproduce the large exciton diffusion
lengths previously reported (J. McNeill, 2000 JCP).  Be suspicious if
anybody reports anything above 15 nm for a conjugated polymer or
disordered material (such as 2010-2011 Barbara papers, older papers by
Gregg, Buratto, etc).  The relationship between diffusion length
$L_D$, the diffusion constant $D$, and donor lifetime $\tau_d$ is
defined by,
%
$$
L_D = \sqrt{2 D \tau_d},
$$
%
which can be easily rearranged to give $D$ from the
experimentally-determined $L_D$ and $\tau_d$.

Typically the parameters we plug into a simulation are the diffusion
length $L_D$, the F\" orster radius $R_0$, the particle size, the
pure donor fluorescence lifetime $\tau_d$, particle size $R_{np}$
and the time step $\Delta t$, typically set to be at least a factor of
20-50 smaller than the lifetime of the donor.  The basic physical
picture then is that we have an exciton that is generated at some
random position within a nanoparticle.  The exciton hops from site to
site within the sphere of the nanoparticle over many time steps, and
at each time step it can undergo decay or energy transfer.  The
exciton continues on its trajectory until decay or energy transfer
occurs.

Now let's get down to the problem of calculating trajectories.  At
each time step, there is some probability that the exciton will decay
by either fluorescing or undergoing energy transfer. We need an
expression for calculating the probability of fluorescence decay per
time step.  The probability of a given exciton decaying per time step
is given by,
%
$$
p=1-e^{-\Delta t/\tau_d} \simeq \frac{\Delta t}{\tau_d}.
$$
%
So, at each time step (each iteration in a loop), we generate a random
number between 0 and 1, and compare it to the above probability to
determine if the exciton decays.

Now we need an expression for the probability of FRET occurring during
a given timestep.  We know that the rate of energy transfer $k_{ET}$
is given by the expression,
%
$$
k_{ET} = \frac{1}{\tau_D} \left( \frac{R_0}{R} \right)^6
$$
Thus the probability of energy transfer at a given time step is given by,
%
$$
p_{ET} = \frac{\Delta t}{\tau_D} \left( \frac{R_0}{R} \right)^6
$$
%
Since there will typically be multiple acceptors in a nanoparticle, in
our simulation we will need to sum up the rates of energy transfer of
a given exciton to each acceptor to give a total energy transfer rate.

\vskip 24 pt

\noindent {\Large \bf The structure of the {\tt etdiffnp} simulation
  code} \vskip 10 pt

In order to properly simulate this exciton hopping process using a
Monte Carlo approach, we need to run lots and lots of randomly
generated trajectories and collect information about what occurred and
when for each exciton trajectory.  In addition, we will average over
many possible random arrangements of dye molecules, since the
arrangement of dye molecules will vary from particle to particle and
that will affect the results.  There are 3 {\sc matlab} scripts that
do the work: {\tt etdiffnp.m}, {\tt etdiffnp\_hlp.m}, and {\tt
  etdiffnp\_fun.m}.

{\tt etdiffnp.m} is the script that sets up the system physical
parameters and simulation parameters, runs the other scripts that do
the actual simulations, and saves and plots the simulation results.
This is the script that you will have to edit.

{\tt etdiffnp\_hlp} is the script that does some additional setup (mostly
generating the random dye positions) and calls {\tt etdiffnp\_fun},
which calculates the exciton trajectories by a 3D random walk.

To run the simulations, edit {\tt etdiffnp.m} and change parameters to
match the parameters for the system you want to simulate.  You'll
probably want to put the simulation scripts in a directory like {\tt
  c:$\backslash$diffmodel}.  Then start {\sc MATLAB}, type {\tt cd
  c:$\backslash$diffmodel} and then run {\tt etdiffnp}.  After about
5-10 minutes, the simulation should end, and the donor fluorescence
lifetime simulation results will be plotted.  The time variable is
{\tt tt}, the fluorescence intensity is in {\tt fdecay}, and the
energy transfer efficiency is in {\tt EffVec}.  Several useful numbers
are also printed to the command window, including the quenching
efficiency, its uncertainty, and the mean exciton lifetime.  I will
try to add a fit to find the $\beta$ parameter, as well.

\vskip 24 pt

\noindent {\Large \bf Notes on simulations of non-exponential
  dynamics} \vskip 10 pt

These are miscellaneous notes regarding the simulations.  We will
discuss these after you've digested this document and taken a look at
the code and run a couple of simple simulations.

\begin{enumerate}

\item We need to increase the number of excitons, {\tt nex} in order to get reliable
lifetime information.  Keep increasing {\tt nex} and/or {\tt navg}
until {\tt length(decaylist)} is above 20000.  Maybe closer to 100,000
will be required to get the noise level down to a level where we can
get acceptable fits to find the stretch parameter.  We can also edit
the {\tt histran} variable to make the bin sizes larger for the
histograms, but this also worsens the time resolution.

\item Will want to get good lifetime simulations for energy transfer
efficiencies ranging from about 20\% to about 80\%.

\item If 1 quencher per particle gives a quenching efficiency that is
  higher than you need (say, you want 20\% but it gives 40\%), then
  increase the particle size, or decrease the Forster radius or
  diffusion length, depending on what makes more sense.  If you are
  not sure, then just increase the particle size.

\item Once we get the decays, we will want to fit them to a
  ``stretched exponential'' (actually Wikipedia has a nice intro on
  this function) in order to try to determine the ``stretch''
  parameter $\beta$, and how the average lifetime and the stretch
  parameter are related to the average energy transfer efficiency.

\item I'm not sure if the code has the right particle size, diffusion
  length, particle radius that correspond to the PFBT nanoparticles
  you have been making.  Hopefully you can obtain the correct numbers
  from JJ's dissertation or Jiangbo's paper (run them by me as well).

\item The amount of time it takes to do a simulation is very sensitive
  to {\tt dt}.  It might be best to start with a relatively large {\tt
    dt} like 10 or 20 ps, and a small {\tt nex=1500}, try a bunch of
  different {\tt ndye} values until you get efficiency that is about
  right, then once you have found the right {\tt nyde}, reduce {\tt
    dt} to 5, 2 or even 1 ps, to improve accuracy and time resolution
  at a cost of increased simulation time), and also crank up {\tt nex}
  and {\tt navg} to make the decays less noisy.  In other words, a
  larger time step and fewer excitons or averages gets you fairly
  accurate efficiency results fast, so you can figure out how many
  quenchers you want to use. After that, for more accurate dynamics,
  you need to increase the number of excitons and averages, and reduce
  the time step.

\item Be careful not to confuse radius with diameter.  It's an easy
  mistake to make.

\item You might want to first do some semilog plots and check for
deviations from exponential behavior.

\item I have some fitting code called {\tt theofit}.  To use it on
  some simulation results, change to the appropriate directory and run
  {\tt theofitrun}.  The parameters are $\tau_{kww}$ and $\beta$,
  respectively.  $\tau_{kww}$ is usually a number between 20 and 300
  (picoseconds), and the stretch parameter $\beta$ should be a number
  between 0.3 and 1.

\item Further down the to-do list is to try to estimate what the
  acceptor dynamics should look like.  There should be an exponential
  or stretched exponential rise.  We should work through the
  appropriate rate equations for the simple case and also simulate
  what the data would look like, to see if we have a chance of
  possibly getting out some useful information from TCSPC measurements
  of the rise of the acceptor emission.

\end{enumerate}

\vskip 24 pt

\noindent {\Large \bf Append A - Energy Transfer Efficiency as a
  Function of Distance } \vskip 10 pt

For checking the code, I verified that, in the absence of
diffusion, energy transfer to a single quencher gives the proper
curve for energy transfer efficiency (probability) as a function of
distance $R$.  The proper functional form is,
%
$$
p(R) = \frac{1}{1 + \left( \frac{R}{R_0} \right)^6}
$$

\vskip 24 pt

\noindent {\Large \bf Appendix B - Diffusion and Decay With a Point
  Source/Sink } \vskip 10 pt

For checking the code, I verified that it reproduces the correct
behavior for a pointlike sink (a quencher with $R_0$ set to 0.1 nm,
$L_D$ = 3 nm). The functional form for diffusion and decay from a
point source is identical to (or a mirror image of) the functional
form for diffusion and decay for a point sink.  This problem was
perhaps first solved by Fermi in his calculations of neutron diffusion
leading up to the demonstration of the first nuclear fission chain
reaction, and comes up in a number of real world problems, such as
spreading of a radioactive plume, spreading of toxic compounds in an
aquifer, etc.  At steady-state, diffusion-decay from a point source in
3D yields a density of,
%
$$
n(R) = \frac{n_0 K_0(R/L_D)}{R}
$$
%
where $K_0$ is the zeroth-order modified Bessel function and $n_0$ is
the density at the point source.  In {\sc matlab}, this is given by
{\tt besselk(0,R/LD)}.  In 2D, the solution is,
%
$$
n(R) = \frac{n_0 K_0(R/L_D)}{\sqrt{R}},
$$
%
and in 1D, the solution is,
%
$$
n(x) = n_0 K_0(x/L_D).
$$
%

For diffusion-related information, I suggest a web search and literature search
on the key words ``diffusion length'', the Powell papers, and the book
The Mathematics of Diffusion, by J. Crank.

\end{document}
