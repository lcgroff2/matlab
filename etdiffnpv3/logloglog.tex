\documentclass[12pt]{article}
\usepackage{graphicx,cite}
\usepackage{times}
\usepackage[letterpaper]{geometry}
\usepackage{amsmath}
%\usepackage{doublespace}
\geometry{text={6.75in,9in}, top=0.6in,
left=0.75in}

\newcommand{\rang}      {\AA^{-1}}      %reciprocal angstroms


\def\bra#1{\langle#1\vert}              % \bra{stuff} gives <stuff|
\def\ket#1{\vert#1\/\rangle}            % \ket{stuff} gives |stuff>
\def\vev#1{\langle{#1}\rangle}          % \vev{stuff} gives <stuff>
\def\Ham{{\mathcal H}}
\def\D{{\partial}}
\def\DD#1#2{\frac{\partial#1}{\partial#2}}
\def\EE#1{\times 10^{#1}}


\title{Fitting Stretched Exponentials}
\date{April 4, 2008}

\begin{document}
\maketitle

The stretched exponential or Kohlrausch-Watts-Williams (function)
often describes dynamics in heterogeneous systems characterized by a
range of rates.  It is of the form,
%
$$
I = I_0 e^{-(t/\tau)^\beta}
$$
%
The $\beta$ parameter is the ``stretch'' parameter, and it varies
between 0 and 1.  A $\beta$ parameter of 1 is just an ordinary
exponential decay.  So, the lower the $\beta$ parameter is, the more
heterogeneous the dynamics.  Therefore $\beta$ obtained from fits can
act as a quantitative measure of heterogeneity.  We will attempt to
extract $\beta$ values from our simulations and fits to determine how
the heterogeneity in the fluorescence dynamics depends on various
parameters such as particle size, exciton diffusion length, F\" orster
radius, and number of dyes.  Let's play with the math and see if we
can come up with a way to do a straight-line fit that will give us
$\beta$ as the slope.  If we divide both sides by $I_0$, and take the
natural log of both sides, we get,
%
$$
\ln \left( \frac{I}{I_0} \right) = - \left( \frac{t}{\tau}
\right)^\beta
$$
%
This function is of the form,
%
$$
y = a x^b
$$
%
If we perform a log-log plot of $y$ versus $x$, we obtain a straight
line with a slope of $b$.  If you want a little more detail and
example plots, see ``log log plot'' on Wikipedia.  The $y$ value at
$x=1$ gives us the $a$ parameter.

Mapping this onto the KWW form, a plot of $\ln (- \ln I/I_0)$ versus
$\ln t$ should yield a straight line of slope $\beta$.  The intercept
should tell us something about $\tau$.

Let's try this out on some simulated KWW-type data in {\sc matlab}.
First we set up our time axis, $\tau$, and $\beta$:
%
\begin{verbatim}
>> t = 0:200;
>> tau = 30;
>> beta = 0.76;
\end{verbatim}
%
Now we make some simulated fluorescence data,
%
\begin{verbatim}
>> f=exp(-(t/tau).^beta);
\end{verbatim}
%
Notice we used \verb=.^= since we are raising an array to a power.
Now let's do a log-log plot to see if it looks like a straight line,
%
\begin{verbatim}
>> loglog(t,-log(f/f(1)))
\end{verbatim}
%
Yes, it looks like a straight line.  Recall that in {\sc matlab} {\tt
log} means natural (base $e$) log, while {\tt log10} is the base 10
logarithm.  Now we have to do a linear fit in
order to extract the $\beta$ parameter.  This is done as follows.
First we need to take the log-log of the fluorescence intensity and
the log of the time axis.  We will place the results in {\tt loglogf}
and {\tt logt}.
%
\begin{verbatim}
>> loglogf=log(-log(f/f(1)));
>> logt = log(t);
\end{verbatim}
%
Now we fit to a straight line,
%
\begin{verbatim}
>> pp=polyfit(logt,loglogf,1)


\end{verbatim}
% 
This yielded a bad fit, because the first point of {\tt loglogf} is
either {\tt NaN} or {\tt -Inf}.  Since the first point is {\tt
f(1)/f(1)} which is 1, the log of that is zero.  Then we take the log
of zero, and get minus infinity.  We need to remove the first point and
fit the rest:
%
\begin{verbatim}
>> loglogf=loglogf(2:end);
>> logt=logt(2:end);
>> pp=polyfit(logt,loglogf,1)
pp =

   0.76000  -2.58491

\end{verbatim}
%
That fixed it!  The first parameter, which is the slope, is 0.76, which is the
$\beta$ parameter we started with.  The intercept is somehow related
to $\tau$, but we won't worry about that for now.  We might want to
check the fit by plotting:
\begin{verbatim}
>> plot(logt,loglogf,'o',logt,polyval(pp,logt))
\end{verbatim}
%
I have put all this together in a little m-file {\tt logloglog.m}
which hopefully might help us to analyze our simulation results, etc.


\end{document}
